\chapter{Product information}
%Reducing the spacing from the title
\vspace{-6em}

\newglossaryentry{PdtName}{name={HighwayToSafety}, desciption={Name of our
Crisis Management System}}

\section{Identification}
%Include precise information of the software product like identification name
%(that you can include in the \gls{PdtName}), list of parts that compose it
%(indicating identification numbers for each part).
%pecify the applicable operating environment(s), including version(s) of
% hardware, communications, and operating system(s).
<\gls{PdtName} - XN1000> is a web application meant to be used on any internet
browsers on any platform and an adapted version is also included for any iPad
with a version of iOS later than iOS 7.

\section{Copyright}
Copyright \copyright \ 2016 by University of Luxembourg. All rights reserved.

\section{Trademark notices}
HighwayToSafety \circledR, the HighwayToSafety Logo and related trade dress are
trademarks or registered trademarks of the University of Luxembourg and/or its
affiliates, and may not be used without written permission. All other trademarks
are the property of their respective owners. University of Luxembourg is not
associated with any product or vendor mentioned in this user-manual.

\section{Restrictions}
No part of this manual, including the software described in it, may be
reproduced, transmitted, transcribed, stored in a retrieval system or translated
into any language in any form or by any means, except by the purchaser for
backup purposes, without prior written permission of the University of
Luxembourg.

\section{Warranties}
University of Luxembourg warrants that for a period of two years from the date
of purchase that the Software conforms to it's published specifications. This
limited warranty extends only to Customer as the original licensee. In case of
malfunctioning and not providing the mentioned specification, University of
Luxembourg offers to refund or replace the software to their customers. In no
event does University of Luxembourg warrant that the Software is error free or
that Customers will be able to operate the Software without problems or
interruptions. This warranty does not apply if the software (a) has been
altered, except by University of Luxembourg, (b) has not been installed,
operated, repaired, or maintained in accordance with instructions supplied by
University of Luxembourg, (c) has been subjected to abnormal physical or
electrical stress, misuse, negligence, or accident, or (d) is used in
ultrahazardous activities.

\section{Contractual obligations}
All information, written or oral, that the customer discloses or makes available
to the developer through any means of communication is confidential.
University of Luxembourg performs services for the customer such as support and
maintain the software during the validity of the warranty.
University of Luxembourg will be responsible for conducting all activities
required to install the software at the customer's premises.
The customer will have 10 days following the date of installation to test the
software.
If University of Luxembourg fails to deliver the software with the desired
specifications, the customer shall detail in writing its reasons for rejection.
In that case, University of Luxembourg shall correct the software.

\section{Disclaimers}
University of Luxembourg makes no representations or warranties, either
expressed or implied, with respect to the contents hereof and specifically
disclaims any warranties, merchantability or fitness for any particlar purpose.
Further, University of Luxembourg reserves the right to revise this publication
and to make changes from time to time in the contents hereof without the
obligation of this institution to notify any person of such revision or changes.

\section{Contact}
In case of any problem or question the customer can visit our website:
https://uni.lu/highwaytosafety.                                                 
They can also reach our helpline during business hours: 00352 46 12 34 56.      
Or send an email to: highwaytosafety.help@uni.lu