\chapter{Introduction}
\label{chap:introduction}

\section{Scope}
This section has to provide the scope of the user's manual document.
In the following some opening statements to use when providing the
information corresponding to this section.

This document provides \ldots
%Example: This document provides minimum acceptable information for knowing how
% to use the software system \mysystemname.


This document does not \ldots 
 
This document is not \ldots
%Example: This document is not intended to provide information about how to
% connect, deploy, configure, or use any external device or
% third-party software system that is rqeuired for the correct funcitoning of
% \mysystemname.

 
This document may be used with \ldots
%This document may be used with other documents provided by third-party
% companies to have an overall view and correct understanding of the environment
% and procedures where the software system \mysystemname is aimed to be deployed
% and run.




\section{Purpose}
In this section you explain the purpose (i.e. aim, objectives) of the user's
manual. In the following some examples of opening statements to be used in this
section.

The purpose of this document is \ldots

This document defines \ldots

This document is meant to \ldots



\section{Intended audience}
Description of the categories of persons targeted by this document together with the description of how they are expected to exploit the content of the document.


\section{mysystemname}
Brief overview of the software application domain and main purpose.


\subsection{Actors \& Functionalities}
Overview of all the \textbf{\emph{\glspl{actor}}} interacting with the software
being them either humans (called end-users in the standard
\cite{IEEE-2001-userdocumentation}) or not. For each actor, describe the main
software functions that are offered to him. Structure of this sub-section MUST
be by actor/functionalities.


\subsection{Operating environment}
%Brief overview of the infrastructure on which the software is deployed and
% used.
The department of Fire and Emergeny Services is a critical part of the State
Services that needs quick actions and communication in order to minimize the
casualties and inconveniences during emergencies such as car crash, school fire,
etc. \\\\
For the sake of a smooth execution of its tasks, it requires the
cooperation of several other infrastucture such as the Hospital departments
which are also greatly impacted during emergencies to take care of the injured
and a good communication service between these institutions would also be
required. \\\\
In our scenarios, it most likely also requires to have the cooperation of
the towing service to take care of the broken cars and the Highway maintenance
service to eventually seal off parts of the Highway.

\section{Document structure}  
%Information on how this document is organised and it is expected to be
%used. Recommendations on which members of the audience
%should consult which sections of the document, and explanations about the used
%notation (i.e. description of formats and conventions) must also be provided.
This user-manual is basically split into two parts. \\\\
The first two chapters are meant to introduce the user-manual. Any user should
have a look at these chapters to get a good overview of the manual. \\\\
The second part going from chapter 3 to chapter 5 are the more technical
part mostly composed of distinct procedures, operations or problems, meaning
that by reading the description of the concerned section, the user should be
able to identify whether it corresponds to their needs. As such, users would
mostly be looking for a procedure in chapter 3 corresponding to their functions
that bears resemblances to their own case and check out the section that
explain in details their part. Otherwise if they encounter any error messages or
problems of the system, they would need to check out chapter 5 for more
information.
Chapter 4 is mainly written for the purpose of understanding how the whole
system works together through different software operation hidden from the
users.





